\documentclass{article}
\usepackage[utf8]{inputenc}
\usepackage[italian]{babel}
\usepackage{graphicx}
\usepackage{hyperref}
\usepackage{cite}
\hypersetup{
    colorlinks=true,
    linkcolor=black,
    filecolor=magenta,      
    urlcolor=cyan,
    pdftitle={Overleaf Example},
    pdfpagemode=FullScreen,
    }

\begin{document}
\begin{titlepage}

    \begin{center}
        \vspace*{1cm}

        \Huge
        \textbf{Bryanair}

        \vspace{0.5cm}
        \LARGE
        Relazione tecnica

        \vspace{1.5cm}

        \textbf{Riccardo Rasori}

        \vfill



        \vspace{0.8cm}

        \includegraphics[width=0.4\textwidth]{iislogo.png}

        \Large
        Informatica e telecomunicazioni\\
        I.I.S. B. Castelli\\
        Brescia\\
        20/04/2024

    \end{center}

\end{titlepage}


\begin{abstract}
    Il mio progetto si pone l'obiettivo di creare un'applicazione web per la prenotazione di voli aerei di una compagnia fittizia chiamata Bryanair \cite{ryanair}. L'applicazione permette di visualizzare i voli disponibili, prenotare un volo, visualizzare i dettagli di un volo e cancellare una prenotazione. L'applicazione è stata sviluppata utilizzando il framework Laravel con lo stack TALL \cite{tall} (TailwindCSS \cite{tailwind}, Alpine.js \cite{alpine}, Laravel \cite{laravel}, Livewire \cite{livewire}) e il linguaggio di programmazione PHP \cite{php} e il DBMS MySQL \cite{mysql} con PHPmyAdmin \cite{phpmyadmin}.
    L'applicazione è stata testata e validata per garantire il corretto funzionamento e la buona esperienza utente. Infine, sono state identificate le best practice nel campo delle prenotazioni online di voli aerei e sono state proposte possibili evoluzioni future per migliorare l'applicazione.

\end{abstract}

\tableofcontents
\section{Introduzione}
\subsection{Presentazione del problema e stato dell'arte}
Il problema affrontato nel progetto riguarda la creazione di un'applicazione web per la prenotazione di voli aerei. Attualmente, molte compagnie aeree offrono servizi di prenotazione online, ma spesso questi sistemi sono complessi da utilizzare o non offrono una buona esperienza utente. L'obiettivo del progetto è quello di creare un'applicazione intuitiva e user-friendly che permetta agli utenti di prenotare voli in modo semplice e veloce. Inoltre, verrà effettuata un'analisi dello stato dell'arte per identificare le soluzioni esistenti e le best practice nel campo delle prenotazioni online di voli aerei.


\subsection{Obiettivo}
L'obbiettivo del progetto è quello di creare un'applicazione web per la prenotazione di voli aerei che sia intuitiva e user-friendly. L'applicazione dovrà permettere agli utenti di visualizzare i voli disponibili, prenotare un volo, visualizzare i dettagli di un volo e cancellare una prenotazione. L'applicazione dovrà essere sviluppata utilizzando il framework Laravel e il linguaggio di programmazione PHP.
\section{Basi teoriche}
\subsection{Architettura MVC}
Una delle basi teoriche fondamentali è il concetto di architettura MVC (Model-View-Controller), che viene utilizzato nel framework Laravel per separare la logica di business (Model), la presentazione (View) e il controllo delle azioni dell'utente (Controller). Questa separazione permette di ottenere un codice più modulare, manutenibile e testabile.


\subsection{DB relazionale}
Un'altra base teorica importante è il concetto di database relazionale. Nel progetto, il framework Laravel utilizza il linguaggio SQL per interagire con il database e gestire i dati relativi ai voli e alle prenotazioni. Il database relazionale permette di organizzare i dati in tabelle e definire relazioni tra di esse, facilitando la gestione e l'accesso ai dati.

\subsection{Autenticazione e autorizzazione}
Un'altra base teorica rilevante è il concetto di autenticazione e autorizzazione. Nel progetto, verranno implementati meccanismi di autenticazione per permettere agli utenti di registrarsi e accedere all'applicazione, e meccanismi di autorizzazione per definire i permessi di accesso alle diverse funzionalità dell'applicazione. Questo garantirà la sicurezza e la privacy dei dati degli utenti.

\subsection{Validazione dei dati e gestione degli errori}
Infine, verranno utilizzati concetti di validazione dei dati di input e gestione degli errori. Nel progetto, il framework Laravel fornisce strumenti per validare i dati inseriti dagli utenti e gestire gli errori in modo elegante e user-friendly. Questo permette di migliorare l'affidabilità e l'usabilità dell'applicazione.
\section{Gestione}
La gestione del progetto è stata organizzata utilizzando il framework Laravel e le sue funzionalità di gestione dei task e delle risorse. Inoltre è stata supportata da strumenti di project management come GitHub per il controllo di versione del codice sorgente. Sono state definite le seguenti fasi di gestione del progetto:
\subsection{Analisi dei requisiti}
In questa fase sono stati identificati i requisiti dell'applicazione, le funzionalità richieste e le aspettative degli utenti. Sono state effettuate interviste e sondaggi per raccogliere informazioni e definire gli obiettivi del progetto.
\subsection{Progettazione}
In questa fase è stata definita l'architettura dell'applicazione, la struttura del database e i flussi di navigazione dell'utente. Sono stati creati diagrammi UML e wireframe per visualizzare e comunicare la progettazione dell'applicazione.
\subsection{Sviluppo}
In questa fase sono stati implementati i modelli, le viste e i controller dell'applicazione utilizzando il framework Laravel. Sono state create le migrazioni per definire la struttura del database e sono stati sviluppati i test per verificare il corretto funzionamento dell'applicazione.
\subsection{Test e validazione}

In questa fase sono stati eseguiti test per verificare il corretto funzionamento dell'applicazione e la conformità ai requisiti. Sono stati identificati e risolti eventuali bug e sono state effettuate sessioni di validazione con gli utenti per raccogliere feedback e migliorare l'esperienza utente.
\subsection{Deploy e manutenzione}

In questa fase l'applicazione è stata deployata su un server di produzione e resa disponibile agli utenti. Sono state implementate funzionalità di monitoraggio e logging per monitorare le prestazioni e la stabilità dell'applicazione. Sono state anche pianificate attività di manutenzione per gestire eventuali aggiornamenti e miglioramenti futuri.



\section{Conclusioni}
In conclusione, il progetto ha permesso di creare un'applicazione web per la prenotazione di voli aerei che è intuitiva e user-friendly. L'applicazione è stata sviluppata utilizzando il framework Laravel e il linguaggio di programmazione PHP, seguendo i principi dell'architettura MVC e utilizzando un database relazionale per gestire i dati. Sono stati implementati meccanismi di autenticazione e autorizzazione per garantire la sicurezza e la privacy dei dati degli utenti, e sono stati utilizzati strumenti per validare i dati di input e gestire gli errori in modo elegante e user-friendly. L'applicazione è stata testata e validata per garantire il corretto funzionamento e la buona esperienza utente. Infine, sono state identificate le best practice nel campo delle prenotazioni online di voli aerei e sono state proposte possibili evoluzioni future per migliorare l'applicazione.
\bibliography{biblio}

\bibliographystyle{acm}
\end{document}